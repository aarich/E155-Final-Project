
\documentclass[12pt]{article}
\usepackage[english]{babel}
\usepackage[utf8x]{inputenc}
\usepackage{amsmath}
\usepackage{graphicx}
\usepackage{booktabs}
\usepackage{float}
\usepackage[colorinlistoftodos]{todonotes}
\usepackage{epstopdf}
\usepackage[margin=1.0in]{geometry}
\usepackage{hyperref}

% Source code formating
\usepackage{listings}
\lstset{ 
 %frame=single,
 language=Verilog,
 %numbers=left,
 breaklines=true,
  basicstyle=\small, %or \small or \footnotesize etc.
 title=\lstname
}


\begin{document}

\begin{titlepage}

\newcommand{\HRule}{\rule{\linewidth}{0.5mm}} % Defines a new command for the horizontal lines, change thickness here

\center % Center everything on the page


\textsc{\LARGE Harvey Mudd College}\\[1.5cm] % Name of your university/college
\textsc{\large Engineering 102}\\[0.5cm] % Minor heading such as course title
\textsc{\Large Advanced Systems Engineering}\\[0.5cm]  % Major heading such as course name

\HRule \\[0.4cm]
{ \huge \bfseries State Space Design of an Inverted Pendulum Controller}\\[0.4cm] % Title of your document
\HRule \\[1.5cm]
 

\begin{minipage}{0.4\textwidth}
\begin{flushleft} \large
\emph{Author:}\\
Aaron \textsc{Rosen}

\end{flushleft}
\end{minipage}
~
\begin{minipage}{0.4\textwidth}
\begin{flushright} \large
\emph{Professors:} \\
John \textsc{Molinder} % Professor's Name

Qimin \textsc{Yang}

Philip \textsc{Cha}
\end{flushright}
\end{minipage}\\[1cm]



{\large Due: May 15, 2015}\\[1cm] 


\begin{abstract}
%THIS NEEDS TO BE COMPLETED WHEN EVERYTHING IS DONE
\end{abstract}

\vfill % Fill the rest of the page with whitespace

\end{titlepage}




%\newpage
%
%{\footnotesize \tableofcontents}

\newpage


%%%%%%%%%%%%%%%%%%%%%%%%%%%%%%%%%%%%
%       INTRODUCTION
%%%%%%%%%%%%%%%%%%%%%%%%%%%%%%%%%%%%
\section{Introduction}
\section{New Hardware}
\subsection{BlueSMiRF}
\subsection{Vehicle}
\section{Schematics}
Alex
\section{Raspberry Pi}
Alex
\subsection{Website}
\subsection{Python/C}
\section{FPGA Design}
\subsection{FPGA}
The FPGA reads data from the BlueSMiRF using UART hardware coded in SystemVerilog, processes and executes the command, and then sends an acknowledgment back to the Raspberry Pi.  It is constructed as a controller-datapath pair with three main submodules in the datapath - receiveMSG, executeCommand, and sendAck.  The SystemVerilog code installed on the FPGA is shown in Appendix C. The FPGA and BlueSMiRF are the only two electrical components on the breadboard. Two motors are connected to the $\mu$Mudd board's H-bridge screw terminals.  The schematic is shown in figure.

The clock used to interface with the BlueSMiRF is implemented using a PLL that oversamples the 115.2 Kbaud UART frequency at 921.6kHz, or a factor of 8.  This oversampler determines if there is an incoming message.  The actual sampling of the BlueSMiRF's TX line is accomplished using a frequency divider that allows for sampling at the correct rate.  The divider's phase can be frozen when the start bit has not yet been detected.  This ensures that the sampling of the line is as close to the center of the transmission's clock as possible.  The Raspberry Pi sends three characters, which are flushed by the Bluetooth module's buffer at the same time, so the command appears as a 30-bit message.  The sampler stops sampling when it sees a stop bit, and either begins a new message if the next bit is a start bit, or signals to the controller that a command has been received of an entire command if the line has remained high.

The FPGA executes the command received by controlling the two motors via the H-Bridges on the $\mu$Mudd board. Each command consists of a PWM setting and rotation direction for each motor and a duration for which the motors should be turning.  A counter is used to create a reference clock for PWM; the power levels are referenced against this counter to determine the correct duty cycle.  Multiplexers are used to route the power to the correct pins on the H-Bridge, allowing for both forward and backwards movement.  To prevent the vehicle from running indefinitely, the timer stops incrementing when the requested duration is reached, and a signal is output that is used to cut power to the motors.  Each LSb of the duration character corresponds to roughly one-tenth of a second.

Once the requested duration has been reached and power to the motors cut, the FPGA transmits the character `A' back to the BlueSMiRF as an ACK code.  After this ACK has been sent, the FPGA will return to the receiveMSG state and start to sample for a new command.

A flowchart detailing each state of the FPGA may be found in Appendix D.

\section{Results}
\section{References}
\section{Parts List}
\section{Appendices}


\subsection{FPGA Code}
\lstinputlisting[language=Verilog]{VehicleControl.sv}

%--------------------------------------------------------------------------------------
%
%\begin{thebibliography}{widest entry}
%
% \bibitem{E102}
%"E102 Final Project," P. Cha et. al., Web. May. 2015.
%
%\end{thebibliography}
%
\vfill

\end{document}
